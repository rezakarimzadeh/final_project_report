

\فصل{برخی نکات نگارشی}

\قسمت{روش‌ پیشنهادی قطعه‌بندی تومور}

قطعه‌بندی تومور در کنار قطعه‌بندی ساختارهای در ریسک، یک نیاز حیاتی است که لازم است با دقت و سرعت بالایی انجام شود تا در روند تشخیص و درمان مورد استفاده قرار گیرد. قطعه‌بندی تومور نیز یک روند زمان‌بر و خسته‌کننده برای شخص متخصص است (به‌خصوص آنکه این قطعه‌بندی در طی درمان باید چندین مرتبه تکرار گردد). بنابراین خودکار کردن این روند در کلینیک‌های درمانی ضروری است. 

قطعه‌بندی تومور به علت تغییرات اندازه، موقعیت و شکل آن، بسیار چالش‌ برانگیزتر از قطعه‌بندی سایر ساختارهای بدن است که شکل و موقعیت خاصی دارند. بر همین اساس این قطعه‌بندی برای روش‌های خودکار نسبت به روش‌های دستی و نیمه-خودکار بسیار پر چالش‌تر است. روش‌های یادگیری عمیق نسبت به سایر روش‌های  خودکار در قطعه‌بندی تومور دقت بالاتری نشان‌ داده‌اند بنابراین  ذز ادامه قصد داریم روشی مبتنی بر روش‌های یادگیری عمیق برای قطعه‌بندی خودکار تومور ارایه دهیم.

\زیرقسمت{معرفی دادگان و پیش‌ پردازش}

قطعه‌بندی تومور مغزی یک مرحله‌ی مهم، برای تشخیص و درمان سرطان مغزی است. تصاویر ام‌آرآی به علت ایجاد تضاد بالا در بافت‌های نرم و رزولوشن‌ بالایی که برای تصویربرداری تامین می‌کند، در تحلیل و بررسی ساختارهای مغزی، مطالعه‌ی تومور مغزی و نظارت و تهیه‌ی نقشه‌ی درمان، کاربرد فراوانی دارد. ام‌آرآی برخلاف روش تصویربرداری سی‌تی اسکن از پرتوهای یونیزه شده استفاده نمی‌کند و از این جهت خطرهای پرتوهای پر انرژی را ندارد. 

در کاربردهای کلینیکی، برای تشخیص بهتر، تصویر‌های ام‌آرآی مکمل سه‌بعدی دیگری از جمله، \lr{T1 contrastenhanced (T1ce)} ،\lr{T2-weighted} ،\lr{T1-weighted} و \lr{Fluid Attenuation Inversion Recover (FlAIR)} برای بهبود تضاد، بین بافت‌های نرم و همچنین تومور و پس زمینه، تهیه می‌شوند \مرجع{myronenko20183d}.

مجموعه دادگان BraTS \LTRfootnote{Brain Tumor Segmentation Dataset} شامل چهار دنباله تصویر ام‌آرآی T1 ، T2 ، T1ce و FlAIR برای هر بیمار است که قطعه‌بندی مطلوب متناظر برای هر بیمار توسط متخصص قطعه‌بندی شده است. در هر قطعه‌بندی سه ناحیه برای تومور مشخص شده است که عبارتند از: هسته‌ی تومور\LTRfootnote{Tumor Core}، ناحیه‌ی پیش‌ توموری\LTRfootnote{Peritumoral Edema} و تومور در حال رشد\LTRfootnote{Enhancing Tumor}. این مجموعه دادگان از 369 مورد تشکیل شده ازت که ابعاد تصویر ام‌آرآی در هر مورد $240*240*155$ است و فاصله واکسلی برابر با $1*1*1$ میلی‌متر مکعب است \مرجع{menze2014multimodal, bakas2017segmentation}. شکل ~\رجوع{شکل:برتسسمپل} چهار دنباله‌ی تصویر و قطعه‌بندی متناظر برای نواحی مختلف تومور را نشان می‌دهد.

\شروع{شکل}[H]
\centerimg{04bratsexp.png}{7cm}
\شرح{چهار دنباله تصویر ام‌آرآی T1 ، T2 ، T1ce و FlAIR برای یک بیمار خاص با قطعه‌بندی متناظر برای تومور (قرمز: هسته ی تومور، زرد: ناحیه‌ی پیش‌ توموری و سبز: تومور در حال رشد)}
\برچسب{شکل:برتسسمپل}
\پایان{شکل}

در این مطالعه، هرچند استفاده از چهار دنباله‌ی تصویر، دقت قطعه‌بندی را افزایش می‌دهد اما تنها از یک دنباله‌ی تصویر FlAIR استفاده گردید. دلیل این‌کار به این علت است که در مراکز درمانی معمولا یک دنباله‌ی تصویر در دسترس است و طبق مطالعات انجام شده دنباله‌ی FlAIR نسبت به سایر دنباله‌ها، تضاد بهتری بین تومور و بافت اطراف فراهم می‌کند \مرجع{bergamino2019comparison}. برای پیش‌پردازش این مجموعه تصاویر ابتدا کناره‌های تصویر به مرکزیت آن بریده شد و ابعاد به $144*176*155$ تغییر کرد. سپس مقدار شدت‌ها به مقدار 90 درصد هیستوگرام تجمعی برای نرمال‌سازی صورت گرفت. برای قطعه‌بندی تومور سه برچسب با یکدیگر ترکیب گردید تا قطعه‌بندی تمام تومور انجام گیرد. در نهایت ده درصد از دادگان به عنوان مجموعه‌ی آزمایش و مابقی برای آموزش شبکه‌ استفاده گردید.

\زیرقسمت{معرفی شبکه‌ی بر اساس ساز و کار توجه برای قطعه‌بندی تومور}

چالش اصلی قطعه‌بندی تومور در تصاویر سه‌بعدی پزشکی، تغییرات اندازه و موقعیت تومور و در برخی موارد تضاد پایین میان تومور و بافت‌های کناری است، که این قطعه‌بندی را دشوار می‌کند. روش پیشنهادی ما برای حل این مشکل استفاده از ساز و کار توجه\LTRfootnote{Attention Mechanism} است که باعث می‌شود شبکه، به ناحیه‌هایی که در آن‌ها احتمال حضور تومور وجود دارد، توجه بیشتر کند. برای این منظور، دو ماژول توجه\LTRfootnote{Attention Module} و قطعه‌بندی\LTRfootnote{Segmentation Module} در این مطالعه‌ ارایه گردید که هدف ماژول توجه یافتن موقعیت و ناحیه‌ای است که در آن تومور حضور دارد و سپس ماژول قطعه‌بند با استفاده از اطلاعات ماژول توجه، قطعه‌بندی تومور را به صورت دقیق انجام می‌دهد. 

روش پیشنهادی ما، بر اساس معماری شبکه‌ی UNet دوبعدی، برای ماژول توجه و قطعه‌بند، پایه‌گذاری شده است. ایده‌ی اصلی این مطالعه در شکل ~\رجوع{شکل:اتنشناورال} نشان داده شده است. این ساختار به صورت سریال عمل می‌کند. در ابتدا ماژول توجه یک قطعه‌بندی تقریبی با نرخ مثبت صحیح\LTRfootnote{True Positive Rate} بالا ایجاد می‌کند که برای آموزش این ماژول از ماسک‌های قطعه‌بندی افزایش\LTRfootnote{Dilated}  یافته استفاده می‌شود که وظیفه‌ی آن یافتن موقعیت تومور با یک تابع هزینه‌ی مناسب است. در گام بعدی ویژگی‌های استخراج شده توسط ماژول توجه، در کنار تصویر اصلی به ماژول قطعه‌بند داده می‌شود و قطعه‌بندی نهایی ایجاد می‌گردد که برای آموزش این ماژول از ماسک‌های قطعه‌بندی اصلی (بدون افزایش) استفاده می‌شود.

\شروع{شکل}[H]
\centerimg{04overallattention.png}{10cm}
\شرح{کلیت روش پیشنهادی برای قطعه‌بندی تومور بر اساس ساز و کار ماژول توجه و قطعه‌بند}
\برچسب{شکل:اتنشناورال}
\پایان{شکل}

\زیرزیرقسمت{ماژول توجه}

همانطور که اشاره گردید شبکه‌ی پایه‌ی مورد استفاده، هم برای ماژول توجه و هم برای ماژول قطعه‌بندی، شبکه‌ی UNet می‌باشد که با جزییات روند آموزش در شکل ~\رجوع{شکل:اتنشنارک} نشان داده شده است. هدف ماژول توجه ارایه‌ی یک قطعه‌بندی به صورت تقریبی و یافتن موقعیت تومور است که نرخ مثبت صحیح بالایی داشته باشد و کل تومور را در بر بگیرد. برای این منظور ماسک‌های قطعه‌بندی افزایش یافته به عنوان قطعه‌بندی مطلوب در روند آموزش این ماژول استفاده گردید. علاوه بر این برای اطمینان از بالا بودن نرخ مثبت صحیح، یک تابع هزینه‌ی جدید معرفی گردید که هدف آن بالا بردن این نرخ باشد. 

\شروع{شکل}[H]
\centerimg{04attentionnetArchitecture.png}{17cm}
\شرح{چارچوب پیشنهادی ماژول‌های توجه (قسمت نارنجی) و قطعه‌بند (قسمت سبز) برای قطعه‌بندی تومور به صورت دوبعدی. از معماری UNet به عنوان شبکه‌ی پایه استفاده گردید.}
\برچسب{شکل:اتنشنارک}
\پایان{شکل}

معادله‌ی این تابع هزینه را می‌توان به صورت زیر نوشت که در آن TP ، مثبت صحیح، FN ، منفی غلط، TN ، منفی صحیح و FP ، مثبت غلط است. 

\begin{alignat}{5}
	&SensitivityLoss = 1-\frac{TP}{TP+FN}  \notag && \\
	&SpecificityLoss = 1-\frac{TN}{TN+FP} \notag && \\
	&AttentionLoss = SensitivityLoss + \alpha \times SpecificityLoss \label{attloss} && 
\end{alignat}

وظیفه‌ی این تابع هزینه بیشینه کردن مقدار TP است که با توجه به معادلات این کار در بخش $SensitivityLoss$ انجام می‌شود اما اگر این قسمت به تنهایی استفاده شود خروجی نهایی یک تصویر تمام یک است. برای حل این مشکل و مصالحه میان مثبت صحیح (TP) و منفی صحیح (TN) یک قسمت دیگر ($SpecificityLoss$ ) اضافه گردید تا این مصالحه صورت پذیرد. در نهایت جمع وزن‌دار (با ضریب $\alpha$) این دو قسمت تابع هزینه‌ی نهایی برای ماژول توجه را تشکیل می‌دهد.

\زیرزیرقسمت{ماژول قطعه‌بند}

قسمت سبز رنگ در شکل ~\رجوع{شکل:اتنشنارک}، ماژول قطعه‌بند را نشان می‌دهد. در این قسمت برای آموزش از قطعه‌بندی‌های مطلوب بدون انجام هیچ پردازش (بدون افزایش)، استفاده گردید. برای تابع هزینه‌ی این قسمت از تابع BCE میان پیش‌بینی شبکه و خروجی مطلوب استفاده گردید. 

به قسمت ورودی این ماژول سه تصویر دوبعدی داده می‌شوند که عبارتند از: الف) تصویر دوبعدی ام‌آرآی، ب) خروجی ماژول توجه و پ) ضرب دو مورد قبل در یکدیگر. این سه تصویر در کنار هم قرار می‌گیرند و به ورودی شبکه داده می‌شوند. علاوه بر این ورودی‌ها ویژگی‌های استخراج شده از قسمت Decoder ماژول توجه در کنار ویژگی‌های ماژول قطعه‌بند قرار داده می‌شوند و با اعمال کانوولوشن در Decoder ماژول قطعه‌بند، خروجی قطعه‌بندی نهایی ایجاد می‌گردد. این روند قسمت قطعه‌بند را مجبور به توجه بیشتر به نواحی دارای بافت تومور می کند.

\زیرزیرقسمت{جزییات پیاده‌سازی}

با توجه به شکل ~\رجوع{شکل:اتنشنارک} دو شبکه‌ی مورد استفاده از معماری UNet تبعیت می‌کنند و دارای دو بخش Decoder و Encoder با لایه‌های کانوولوشنی، ادغام و دیکانوولوشن هستند. برای آموزش شبکه‌ها از مجموعه دادگان قطعه‌بندی تومور مغزی در تصاویر ام‌آرآی (BraTS) به صورت دوبعدی با پیش‌پردازش‌های توضیح داده شده در ابعاد $144*176$ استفاده گردید.
 
برای آموزش ماژول توجه، ماسک‌های قطعه‌بندی آموزش، با ضریب 21 افزایش داده‌ شد. از تابع هزینه‌ی ~\رجوع{attloss} با قرار دادن ضریب $\alpha$ برابر با $0.5$ برای تاکید بیشتر در ایجاد نرخ مثبت صحیح، در روند آموزش استفاده گردید. در نهایت این مدل آموزش دیده شد و به عنوان یک مدل از پیش آموزش داده شده در روند آموزش ماژول قطعه‌بندی استفاده شد. 

در طی آموزش ماژول قطعه‌بند، خروجی و ویژگی‌های تولید شده در ماژول توجه برای هر مورد استخراج گردید و به ماژول قطعه‌بند طبق شکل ~\رجوع{شکل:اتنشنارک} داده شد. تابع هزینه‌ی این ماژول از BCE میان خروجی مطلوب و پیش‌بینی شبکه ایجاد گردید. برای بهینه‌سازی وزن‌ها از الگوریتم بهینه‌سازی Adam استفاده گردید. برای مقایسه‌ی نتایج، UNet استفاده شده در قسمت ماژول قطعه‌بندی، یکبار بدون استفاده از ماژول توجه و یکبار بااستفاده از این ماژول آموزش داده شد. در ادامه به بررسی نتایج پرداخته می‌شود.

\زیرزیرقسمت{نتایج}
 
برای ارزیابی عملکرد ماژول توجه، همانطور که اشاره شد، هدف بیشینه کردن نرخ صحیح مثبت است که طبق معادله‌ی زیر این معیار محاسبه می‌شود.

\begin{alignat}{5}
	&TPR = \frac{TP}{TP+FN} \label{tpr} && 
\end{alignat}

این معیار ارزیابی برای مجموعه دادگان آزمایش برابر با $0.998 \pm   0.003$ بدست آمد که نشان می‌دهد موقعیت تومور به طور صحیح و تقریبا کامل پیدا شده است. شکل ~\رجوع{شکل:اتنشناوتس} قسمت (ب) خروجی ماژول توجه را نشان می‌دهد که در مقایسه با قطعه‌بندی مطلوب (قسمت (الف))، تومور را به صورت نادقیق اما با نرخ مثبت صحیح بالا قطعه‌بندی نموده است.

\شروع{شکل}[H]
\centerimg{04attentionouts.png}{16cm}
\شرح{قطعه‌بندی تومور در نماهای مختلف برای یک مورد از دادگان آزمایش، (الف) قطعه‌بندی مطلوب، (ب) خروجی ماژول توجه، (پ) خروجی ماژول قطعه‌بند بر اساس ساز و کار توجه و (ت) خروجی UNet بدون استفاده از ماژول توجه}
\برچسب{شکل:اتنشناوتس}
\پایان{شکل}

برای ارزیابی عملکرد ماژول قطعه‌بند از معیارهای Dice ، Jaccard و Hausdorff استفاده گردید. این معیارها برای دو شبکه‌ی Unet آموزش دیده شده به صورت تکی و با استفاده از ماژول توجه، استخراج گردید. با توجه به جدول ~\رجوع{attable} میانگین معیار Dice با استفاده کردن از مازول توجه از $0.68$ به $0.79$ افزایش یافته است و انحراف معیار آن از $0.18$ به $0.12$ کاهش یافت. برای مقایسه‌ی معنادار بودن این بهبود از معیار P-value استفاده گردید که در سطر آخر جدول نشان داده شده است. کمتر بودن این مقدار از $0.05$ نشانگر پیشرفت چشمگیر است که نشان می‌دهد استفاده از ماژول توجه باعث افزایش دقت قطعه‌بندی می‌شود. شکل ~\رجوع{شکل:باکساتنشن} نمودارهای جعبه‌ای معیارهای ارزیابی را برای دو شبکه نشان می‌دهد. واضح است میانگین و انحراف معیارهای این معیارها با به کارگیری ماژول توجه بهبود می‌یابد.

\begin{table}[H]
	\caption{معیارهای ارزیابی برای مجموعه دادگان آزمایش برای قطعه‌بندی تومور در دو شبکه‌ی Unet بر اساس ماژول توجه و Unet تنها}
	\label{attable}
\begin{tabular}{llll}
	\hline
	\multirow{2}{*}{نام شبکه} & \multicolumn{3}{c}{نام معیار}                     \\ \cline{2-4} 
	& Dice  & Jaccard  & Hausdorff    \\ \hline
	Unet تنها     & $0.68 \pm 0.18$ & $0.57 \pm 0.17$   & $6.6 \pm 1.0 $           \\ \hline
	Unet با ماژول توجه      & $0.79 \pm 0.12$ & $0.69 \pm 0.15$   & $6.2 \pm 1.0$            \\ \hline
	P-value                      & $0.03 $       & $0.02 $         & $0.06$                 \\ \hline
\end{tabular}
\end{table}


\شروع{شکل}[H]
\centerimg{04attentionboxplot.png}{16cm}
\شرح{نمودار جعبه‌ای معیارهای ارزیابی Dice ، Jaccard و Hausdorff برای مجموعه دادگان آزمایش در دو شبکه Unet بر اساس ماژول توجه (AbUnet) و Unet تنها}
\برچسب{شکل:باکساتنشن}
\پایان{شکل}

\زیرزیرقسمت{جمع‌بندی و نتیجه‌گیری}

در این مطالعه، یک ماژول توجه برای قطعه‌بندی تومور مغزی ارایه گردید. در این ماژول یک تابع هزینه‌ی جدید برای بیشینه کردن نرخ صحیح مثبت پیشنهاد شد و برای آموزش آن از قطعه‌بندی‌های افزایش (Dilated) داده شده استفاده گردید. این ماژول، در واقع، یک قطعه‌بندی نادقیق اما در برگیرنده‌ی تمام تومور پیش‌بینی می‌کند و می‌تواند موقعیت تومور را با دقت بسیار بالایی استخراج نماید. 

در نهایت برای ارزیابی این ماژول توجه، یک UNet یک مرتبه با به‌کار گیری این ماژول و بار دیگر به صورت تکی بر روی دادگان یکسان، آموزش دیده شد و با استخراج معیارهای ارزیابی، ملاحظه گردید، با به‌کارگیری این ماژول دقت چشمگیری در قطعه‌بندی نهایی خواهیم داشت.

\قسمت{جمع‌بندی}