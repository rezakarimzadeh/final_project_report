
\فصل{نتیجه‌گیری و کارهای آینده}

در این پایان‌نامه برروی قطعه‌بندی ساختارهای در ریسک و تومور با استفاده از روش‌های یادگیری ماشین و یادگیری عمیق مطالعه صورت گرفت. پیش از ظهور یادگیری عمیق روش‌های بسیار متنوعی برای قطعه‌بندی تصاویر طبیعی و پزشکی ارائه شده بود که در فصل سوم این روش‌ها معرفی شدند. اما با پیدایش روش‌های یادگیری عمیق و با به‌کارگیری آن‌ها در قطعه‌بندی تصاویر پزشکی، دقت قطعه‌بندی نسبت به روش‌های پیشین افزایش یافت و تقریبا تمام روش‌های کلاسیک جای خود را به روش‌های مبتنی بر هوش مصنوعی دادند.

در این مطالعه‌، پس از معرفی ابزارهای موجود در یادگیری عمیق در فصل دوم، اقدام به پیشنهاد روش‌هایی برای قطعه‌بندی ساختارهای در ریسک و تومور گردید که قطعه‌بندی‌ این ساختارها یک گام ضروری و اساسی در پرتودرمانی برای تهیه‌ی نقشه‌ی دوز تابشی به بدن بیمار است و جلوی آسیب به ساختارهای سالم اطراف تومور (ساختارهای در ریسک) را کمینه می‌کند و میزان تابش به بافت تومور را در حالت بیشینه قرار می‌دهد.

در این پایان‌نامه، چهار روش قطعه‌بندی بر اساس ابزارهای  یادگیری عمیق پیشنهاد گردید که در سه روش اول به قطعه‌بندی ساختارهای در ریسک پرداخته شد و در روش آخر قطعه‌بندی تومور مورد مطالعه قرار گرفت. در ادامه روش‌های ارایه شده به طور اجمالی مرور خواهند شد.

\شروع{فقرات}
\فقره \مهم{قطعه‌بندی ساختارهای در ریسک با چارچوب چگالش دانش: }هدف در چارچوب چگالش دانش، انتقال دانش از یک مدل بسیار پیچیده (آموزگار) که دقت بالایی در کنار هزینه‌های محاسباتی بالا به یک مدل ساده‌تر (دانش‌آموز) است که بدون افزایش هزینه‌های محاسباتی، دقتی مشابه با آموزگار پیدا کند. بنابراین با توجه به این چارچوب، یک مدل بسیار پیچیده، با تعداد پارامتر و هزینه‌ی محاسباتی بالا، بر روی دادگان آموزش قطعه‌بندی ساختارهای در ریسک آموزش داده شد. سپس دو مدل ضعیف‌تر که به عنوان دانش‌آموز معرفی شدند، پیاده‌سازی و یک مرتبه بدون چارچوب چگالش دانش و یک مرتبه با چگالش دانش از آموزگار، آموزش دیدند. ملاحظه گردید که با استفاده کردن از این چارچوب بدون افزودن به پیچیدگی مدل و هزینه‌های محاسباتی، دقت آن‌را افزایش داد.

\فقره \مهم{قطعه‌بندی ساختارهای در ریسک با چارچوب بازخورد خطای پیش‌بین: }این روش بر اساس مدل کدگذار پیش‌بین برای مدل‌سازی عملکرد نورون‌های مغزی در تعامل با یکدیگر، پایه‌گذاری شد. در مدل کدگذار پیش بین، نورون‌های لایه‌های بالایی مغز، سعی در پیش‌بینی خروجی نورون‌های سطح پایینی مغزی را دارند و با ارتباط‌های بازخوردی، مقادیر پیش‌بینی شده از لایه‌های بالاتر به لایه‌های پایین‌تر عبور داده‌ می‌شوند. بر همین اساس یک چارچوب پیشنهاد گردید که از یک Encoder و دو Decoder تشکیل شده بود. Decoder اول بازسازی تصویر ورودی و Decoder دوم، قطعه‌بندی ساختارهای در ریسک را انجام می‌دهند. بنابراین با بازخورد مجموع تصویر بازسازی شده و خطای بازسازی به مسیر قطعه‌بندی، نواحی‌ای که اطلاعات ریزتری دارند و در قسمت بازسازی، به صورت دقیق بازسازی نشده‌اند، تاکید بیشتری می‌شود. دیگر مزیت این روش، استفاده از یک Encoder مشترک در بازسازی و قطعه‌بندی است که باعث استخراج ویژگی‌های سطح بالاتر در شبکه می‌شود. استفاده از این چارچوب، به دلایل گفته شده، باعث افزایش دقت در قطعه‌بندی ساختارهای در ریسک گردید.

\فقره \مهم{قطعه‌بندی ساختارهای در ریسک به صورت سه‌بعدی با پیشنهاد یک تابع هزینه بر اساس شکل: }در این مطالعه سعی شد، مساله‌ی عدم توانایی شبکه‌های عصبی عمیق در استخراج ویژگی‌های مربوط به شکل مورد مطالعه قرار گیرد. رویکرد ما در این مطالعه ایجاد یک فضای شکل بر اساس برچسب‌های قطعه‌بندی و استخراج بردارهای ویژه بود. سپس با توجه به وزن‌های بازسازی میان شکل‌های معتبر و نامعتبر تمایز قایل شد و در نهایت بر این اساس یک تابع هزینه برای اجبار شبکه به تولید شکل‌های معتبر و جلوگیری از ایجاد جزیره‌های واکسلی، تکه‌تکه شدن ساختار و وجود سوراخ در قطعه‌بندی ساختار یکپارچه، می‌توان جلوگیری کرد. اکثر ساختارهای بدن در افراد مختلف معمولا یک شکل دارند و می‌توان با آموزش شبکه با این تابع هزینه، از ایجاد شکل‌های نامعتبر در قطعه‌بندی ساختارهای در ریسک جلوگیری کرد و دقت قطعه‌بندی را بهبود داد.

\فقره \مهم{قطعه‌بندی تومور با استفاده از سازوکار توجه در شبکه‌های عمیق: }قطعه‌بندی تومور در روش‌های قطعه‌بندی خودکار به علت متغیر بودن شکل، موقعیت و اندازه‌ی تومور بسیار چالش برانگیز است. بنابراین برای حل این مشکلات در شبکه‌های عصبی عمیق بهتر است از سازوکار توجه استفاده گردد تا توجه شبکه به نواحی‌ای که در آنها احتمال حضور تومور وجود دارد جلب گردد. برای این‌کار در این مطالعه، از دو ماژول توجه و قطعه‌بند استفاده گردید. در ماژول توجه، هدف یافتن موقعت تومور به نحوی که نرخ مثبت صحیح بالایی داشته باشد و کل تومور را پیدا کند. برای این‌کار یک تابع هزینه‌ی جدید معرفی گردید که این خواسته را به خوبی برآورده نمود. سپس خروجی ماژول توجه و ویژگی‌های استخراج شده توسط این ماژول به ماژول قطعه‌بند داده می‌شود که این‌کار باعث اجبار شبکه به توجه نمودن به نواحی شامل تومور خواهد شد. در نهایت ارزیابی‌ها نشان داد استفاده از این ماژول توجه دقت قطعه‌بندی را به صورت چشمگیر افزایش می‌دهد.
\پایان{فقرات}

با وجود روش‌های خودکار ارایه شده بر اساس یادگیری عمیق برای قطعه‌بندی تصاویر پزشکی و علی‌رغم افزایش سرعت و دقت نسبت به روش‌های کلاسیک و دستی (به عنوان مثال برای قطعه‌بندی تصویر سی‌تی‌اسکن سر و گردن یک بیمار توسط متخصص حدودا 3 ساعت زمان نیاز است که در طول درمان با تغییر شکل تومور و ساختارهای اطراف چندین بار باید تکرار شود  \مرجع{qazi2011auto, stapleford2010evaluation}) اما 68 درصد متخصصین هنوز به این روش‌ها اعتماد ندارند و قطعه‌بندی به صورت دستی را ترجیح می‌دهند \مرجع{stapleford2010evaluation}. بنابراین یک مرحله‌ی رابط کاربری به عنوان نظارت متخصص در قطعه‌بندی پس از قطعه‌بندی به صورت اتوماتیک لازم است تا اعتمادسازی صورت گیرد. برای این‌کار نرم‌افزارهای زیادی وجود دارد که می‌توان پیش‌بینی قطعه‌بندی توسط شبکه را در آن وارد نمود و توسط متخصص، پس‌پردازش‌های لازم انجام شود. از جمله‌ی این نرم‌افزارها می‌توان به \lr{ITK Library} ، Osirix  ، \lr{3D Slicer} و ... اشاره نمود.

پیشنهادهایی که برای کارهای آینده در قطعه‌بندی تومور و ساختار‌های در ریسک می توان داد به شرح زیر است:

\شروع{فقرات}
\فقره افزودن الگوریتم‌های یادگیری فعال\LTRfootnote{Active Learning}  برای دریافت بازخورد از متخصص و بهبود دقت مدل قطعه‌بند در هربار استفاده.

\فقره استفاده از روش‌های \lr{Few Shot Learning} به علت کم‌بودن داده‌های قطعه‌بندی پزشکی می‌تواند در بهبود دقت و سرعت یادگیر مفید باشد.

 \فقره تعمیم تابع‌ هزینه‌ی پیشنهاد شده بر اساس شکل، در این مطالعه، برای قطعه‌بندی و در نظر گرفتن چند ساختار به صورت همزمان.
 
\فقره ایجاد یک رابط کاربری و قرار دادن الگوریتم‌های مختلف قطعه‌بندی تومور و ساختارهای در ریسک در کنار یکدیگر و قطعه‌بندی همزمان چندین ساختار و تومور به صورت همزمان.
\پایان{فقرات}





