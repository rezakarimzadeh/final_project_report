

\فصل{مقدمه}
یکی از موثرترین روش‌های معرفی شده در دهه‌های گذشته برای درمان سرطان، رادیوتراپی\زیرنویس{Radiotherapy}	 بوده است. در رادیوتراپی از پرتوهای پرانرژی مانند اشعه‌ی ایکس\زیرنویس{X-ray}	 و گاما\زیرنویس{Gamma}	، با دوز\زیرنویس{Dose}	 بالا برای از بین بردن سلول‌های سرطانی استفاده می‌شود. این تشعشعات پرانرژی با برخورد به بافت هدف، در برخی موارد، باعث تجزیه و نابودی ناگهانی سلول‌های سرطانی می‌شود. اما در اکثر موارد با آسیب رساندن به دی‌ان‌ای\زیرنویس{   \lr{Deoxyribonucleic Acid (DNA)}}	 بافت سرطانی، مانع از تکثیر و درنتیجه نابودی آن می‌شود. بنابراین لازم است این پرتوها تنها به بافت هدف (تومور) برخورد کنند تا از آسیب به بافت‌های سالم جلوگیری شود.




\قسمت{تعریف مسئله}

همانطور که اشاره گردید در رادیوتراپی، پرتوهای پرانرژی بین بافت سالم و بافت سرطانی تفکیکی قایل نمی‌شوند و در صورت تابانده شدن پرتو به بافت‌های سالم، این بافت‌ها نیز دچار آسیب دی‌ان‌ای و در نهایت از بین می‌روند. بنابراین لازم است قبل از رادیوتراپی موقعیت بافت‌های سالم و بافت سرطانی با دقت بالایی تفکیک شود تا از آسیب به سلول‌های سالم و حیاتی جلوگیری شود.

به ساختارهای سالم اطراف تومور که باید از پرتوهای پرانرژی در هنگام رادیوتراپی محافظت شوند، \مهم{ساختارهای در ریسک}\زیرنویس{\lr{Organs At Risks (OAR)} }گفته می‌شود. برای شهود بهتر در تفاوت بین ساختارهای در ریسک و تومور شکل‌~\رجوع{شکل:نمونه ساختار در ریسک} را در نظر بگیرید. در ردیف اول این شکل نمونه‌هایی از ساختارهای در ریسک مغزی موجود در یک تصویر سی‌تی اسکن\زیرنویس{   \lr{Computational Tomography (CT) Scan}} در نمای اکسیال\زیرنویس{\lr{Axial)}} مانند چشم‌ها، عصب‌های بینایی، ساقه‌ی مغزی، گوش‌های میانی و ... نشان دهده شده است و در ردیف پایین نیز، تومور متناظر به نمایش گذاشته شده است.

\شروع{شکل}[hb]
\centerimg{oar-tumor.png}{7cm}
\شرح{سه لایه‌ی مختلف از تصویر سی‌تی اسکن مغزی در نمای اکسیال. ردیف اول: ساختارهای در ریسک مغزی، ردیف دوم: تومور مغزی.}
\برچسب{شکل:نمونه ساختار در ریسک}
\پایان{شکل}

در رادیوتراپی معمولا سه اصل اساسی در نظر گرفته می‌شود:
\شروع{فقرات}
\فقره افزایش دوز پرتوهای تابشی به تومور امکان بالا رفتن کنترل بر روی رشد تومور را میسر می‌کند.
\فقره با بالا رفتن کنترل بر روی تومور درجه‌ی بهبود افزایش می‌یابد زیرا از گسترش بیشتر آن جلوگیری می‌شود.
\فقره کم کردن میزان تابش پرتو به بافت‌های سالم اطراف تومور، عوارض جانبی را کاهش می‌دهد.
\پایان{فقرات}
بنابراین برای بیشینه کردن دوز دریافتی تومور و در عین حال کمینه کردن دوز دریافتی بافت‌های سالم، قطعه‌بندی\زیرنویس{\lr{Segmentation)}} دقیق تومور و ساختارهای در ریسک در تصاویر پزشکی یک گام اساسی و حیاتی است \مرجع{ramkumar2016user}. بنابراین در این پژوهش سعی شده است این قطعه‌بندی که اولین و اساسی‌ترین مرحله در رادیوتراپی است با پیشنهاد روش‌هایی بر اساس یادگیری ماشین و یادگیری عمیق با دقت بالایی انجام شود.
	

\قسمت{اهمیت موضوع}
در رادیوتراپی تهیه‌ی یک نقشه‌ی درمان\زیرنویس{\lr{Treatment Plan)}} که موقعیت تومور و ساختارهای در ریسک را مشخص می‌کند ضروری است که در آن میزان تابش پرتوهای پرانرژی به بافت‌های سالم و تومور تعیین می‌شود. ضرورت وجود این نقشه‌ی درمان به این دلیل است  


\قسمت{ادبیات موضوع}

همان‌طور که ذکر شد مسئله‌ی مسیریابی وسایل نقلیه‌ی ناهمگن صورت عمومی مسئله‌ی فروشنده دوره‌گرد می‌باشد. 
مسئله‌ی فروشنده‌ی دوره‌گرد در حوزه‌ی مسائل ان‌پی-سخت\پاورقی{NP-hard} قرار می‌گیرد و با فرض $P \neq NP$ الگوریتم دقیق با زمان چندجمله‌ای برای آن وجود ندارد. بنابراین برای حل کارای این مسائل از الگوریتم‌های تقریبی\پاورقی{Approximation Algorithm}  استفاده می‌شود.

مسئله‌ی فروشنده‌ی دوره‌گرد در حالتی که تنها یک فروشنده در گراف حضور داشته باشد، دو الگوریتم تقریبی معروف دارد.
در الگوریتم اول با دو برابر کردن درخت پوشای کمینه\پاورقی{Minimum Spanning Tree} و میانبر کردن\پاورقی{Shortcut} دورهای بدست آمده، الگوریتمی با ضریب تقریب 2 ارائه می‌شود.
در الگوریتم دوم که متعلق به کریستوفایدز\پاورقی{Christofides}~\cite{Christofides} است، به کمک ساخت دور اویلری\پاورقی{Eulerian Cycle} بر روی اجتماع یال‌های درخت پوشای کمینه و یال‌های تطابق کامل کمینه\پاورقی{Minimum Perfect Matching} از گره‌های درجه‌ی فرد همان درخت، و میانبر کردن این دور، ضریب تقریب $1.5$  ارائه می‌شود.
با گذشت حدود ۴۰ سال از ارائه‌ی این الگوریتم، تا کنون 
ضریب تقریب بهتری برای این مسئله پیدا نشده است.

اخیراً با بهره‌گیری از روش کریستوفایدز و بسط آن برای مسئله‌ی فروشنده‌ی دوره‌گرد چندگانه‌ی همگن (در این حالت از مسئله تعداد فروشنده‌ها در گراف بیش از یکی است و هزینه‌ی پیمایش یال‌ها برای همه‌ی عوامل یکسان است) ضریب تقریب $1.5$ ارائه شده است~\cite{Xu}. در روش مطرح شده بعد از به دست آوردن درخت‌های پوشای کمینه برای هر انبار، به جای استفاده از روش دو برابر کردن یال‌ها، روش کریستوفایدز اعمال می‌شود. به راحتی می‌توان نشان داد که صرف اعمال الگوریتم کریستوفایدز به هر یک از درخت‌های بدست آمده، ضریب تقریب  $1.5$ را بدست نمی‌دهد. بنابراین در روش مذکور، الگوریتم کریستوفایدز روی کل جنگل بدست آمده اعمال می‌شود. نشان داده شده است که با استفاده از یک سیاست جایگزینی مناسب بین یال‌هایی که در جنگل کمینه، موجود هستند و آن‌هایی که در این مجموعه حضور ندارند و اعمال کریستوفایدز روی این جنگل‌ها، می‌توان جوابی تولید کرد که بدتر از $1.5$ برابر جواب بهینه نباشد.


همان‌طور که گفته شد نسخه‌ی ناهمگن این مسئله کمتر مورد توجه قرار گرفته است. در گونه‌ی ناهمگن، بیش از یک عامل (فروشنده) در اختیار داریم که در شروع، هر یک از آن‌ها در گره‌های مجزایی که با عنوان انبار معرفی می‌شوند قرار دارند و هزینه‌ی پیمایش یال‌ها برای هریک از عوامل می‌تواند متفاوت از سایر عامل‌ها باشد. در صورتی که تعداد انبارها $m$ فرض شود از جمله کارهای انجام شده در این مورد ارائه ضریب تقریب $4m$ به کمک حل برنامه‌ریزی خطی تعدیل شده\پاورقی{Linear Programming Relaxation}  و ساخت درخت پوشای کمینه~\cite{4m}، ضریب تقریب $1.5m$ به کمک حل تعدیل برنامه‌ریزی خطی با روش بیضی\پاورقی{Ellipsoid Method} و اعمال الگوریتم کریستوفایدز~\cite{1.5m} و ضریب تقریب 2 به کمک راه حل اولیه-دوگان\پاورقی{Primal-Dual} می‌باشد، روش اولیه-دوگان تنها برای حالتی که دو عامل وجود دارد و هزینه‌ی پیمایش یال‌ها برای یک عامل بیشتر از عامل دیگر باشد مطرح شده است~\cite{Primal_Dual}. 


\قسمت{اهداف تحقیق}

در این پایان‌نامه سعی می‌شود که مسئله‌ی مسیریابی وسایل نقلیه برای زیرگراف‌های ناهمگن مختلف مورد مطالعه قرار گیرد. از جمله زیرگراف‌های مورد نظر ما دور، درخت و مسیر می‌باشد. بعد از مطالعه‌ی کارهای انجام شده در این زمینه سعی می‌شود که مسئله به صورت دقیق‌تر مورد بررسی قرار گیرد.

\قسمت{ساختار پایان‌نامه}

این پایان‌نامه شامل پنج فصل است. 
فصل دوم دربرگیرنده‌ی تعاریف اولیه‌ی مرتبط با پایان‌نامه است. 
در فصل سوم مسئله‌ی دورهای ناهمگن و کارهای مرتبطی که در این زمینه انجام شده به تفصیل بیان می‌گردد. 
در فصل چهارم نتایج جدیدی که در این پایان‌نامه به دست آمده ارائه می‌گردد. در این فصل، مسئله‌ی درخت‌های ناهمگن در چهار شکل مختلف مورد بررسی قرار می‌گیرد. سپس نگاهی کوتاه به مسئله‌ی مسیرهای ناهمگن خواهیم داشت. در انتها با تغییر تابع هدف، به حل مسئله‌ی کمینه کردن حداکثر اندازه‌ی درخت‌ها می‌پردازیم.
فصل پنجم به نتیجه‌گیری و پیش‌نهادهایی برای کارهای آتی خواهد پرداخت.
