

\فصل{مقدمه}
یکی از موثرترین روش‌های معرفی شده در دهه‌های گذشته برای درمان سرطان، رادیوتراپی\LTRfootnote{Radiotherapy}	 بوده است. در رادیوتراپی از پرتوهای پرانرژی مانند اشعه‌ی ایکس\LTRfootnote{X-ray}	 و گاما\LTRfootnote{Gamma}	، با دوز\LTRfootnote{Dose}	 بالا برای از بین بردن سلول‌های سرطانی استفاده می‌شود. این تشعشعات پرانرژی با برخورد به بافت هدف، در برخی موارد، باعث تجزیه و نابودی ناگهانی سلول‌های سرطانی می‌شود. اما در اکثر موارد با آسیب رساندن به دی‌ان‌ای\LTRfootnote{\lr{Deoxyribonucleic Acid (DNA)}} بافت سرطانی، مانع از تکثیر و درنتیجه نابودی آن می‌شود. بنابراین لازم است این پرتوها تنها به بافت هدف (تومور) برخورد کنند تا از آسیب به بافت‌های سالم جلوگیری شود \مرجع{nagpal2019basics}. 




\قسمت{تعریف مسئله}

همانطور که اشاره گردید در رادیوتراپی، پرتوهای پرانرژی بین بافت سالم و بافت سرطانی تفکیکی قایل نمی‌شوند و در صورت تابانده شدن پرتو به بافت‌های سالم، این بافت‌ها نیز دچار آسیب دی‌ان‌ای و در نهایت از بین می‌روند. بنابراین لازم است قبل از رادیوتراپی موقعیت بافت‌های سالم و بافت سرطانی با دقت بالایی تفکیک شود تا از آسیب به سلول‌های سالم و حیاتی جلوگیری شود.

به ساختارهای سالم اطراف تومور که باید از پرتوهای پرانرژی در هنگام رادیوتراپی محافظت شوند، \مهم{ساختارهای در ریسک}\LTRfootnote{\lr{Organs At Risks (OAR)} }گفته می‌شود. برای شهود بهتر در تفاوت بین ساختارهای در ریسک و تومور شکل‌~\رجوع{شکل:نمونه ساختار در ریسک} را در نظر بگیرید. در ردیف اول این شکل نمونه‌هایی از ساختارهای در ریسک مغزی موجود در یک تصویر سی‌تی اسکن\LTRfootnote{   \lr{Computational Tomography (CT) Scan}} در نمای اکسیال\LTRfootnote{\lr{Axials}} مانند چشم‌ها، عصب‌های بینایی، ساقه‌ی مغزی، گوش‌های میانی و ... نشان داده شده است و در ردیف پایین نیز، تومور متناظر به نمایش گذاشته شده است.

\شروع{شکل}[hb]
\centerimg{oar-tumor.png}{7cm}
\شرح{سه لایه‌ی مختلف از تصویر سی‌تی اسکن مغزی در نمای اکسیال. ردیف اول: ساختارهای در ریسک مغزی، ردیف دوم: تومور مغزی \مرجع{grand}.}
\برچسب{شکل:نمونه ساختار در ریسک}
\پایان{شکل}

در رادیوتراپی معمولا سه اصل اساسی در نظر گرفته می‌شود:
\شروع{فقرات}
\فقره افزایش دوز پرتوهای تابشی به تومور امکان بالا رفتن کنترل بر روی رشد تومور را میسر می‌کند.
\فقره با بالا رفتن کنترل بر روی تومور درجه‌ی بهبود افزایش می‌یابد زیرا از گسترش بیشتر آن جلوگیری می‌شود.
\فقره کم کردن میزان تابش پرتو به بافت‌های سالم اطراف تومور، عوارض جانبی را کاهش می‌دهد.
\پایان{فقرات}
بنابراین برای بیشینه کردن دوز دریافتی تومور و در عین حال کمینه کردن دوز دریافتی بافت‌های سالم، \مهم{قطعه‌بندی}\LTRfootnote{\lr{Segmentation}} دقیق تومور و ساختارهای در ریسک در تصاویر پزشکی یک گام اساسی و حیاتی است \مرجع{ramkumar2016user}. بنابراین در این پژوهش سعی شده است این قطعه‌بندی که اولین و اساسی‌ترین مرحله در رادیوتراپی است با پیشنهاد روش‌هایی بر اساس یادگیری ماشین\LTRfootnote{\lr{Machine Learning}} و یادگیری عمیق\LTRfootnote{\lr{Deep Learning}} با دقت بالایی انجام شود.
	

\قسمت{اهمیت موضوع}
در رادیوتراپی تهیه‌ی یک نقشه‌ی درمان\LTRfootnote{\lr{Treatment Plan}} که موقعیت تومور و ساختارهای در ریسک را مشخص می‌کند ضروری است که در آن میزان تابش پرتوهای پرانرژی به بافت‌های سالم و تومور تعیین می‌شود. ضرورت وجود این نقشه‌ی درمان به این دلیل است تا تابش حداکثری به بافت تومور و کمینه کردن آسیب به بافت‌های سالم صورت گیرد و در نهایت با تعیین توزیع پرتو، بین بافت سالم و بافت سرطانی می‌توان اطمینان یافت که سلول‌های سرطانی از بین رفته‌اند و میزان آسیب به ساختارهای در ریسک به چه اندازه بوده است \مرجع{wang2019artificial}.

برای تهیه‌ی یک نقشه‌ی درمان، اولین مرحله، قطعه‌بندی تومور و ساختار‌های در ریسک است. به طور معمول این قطعه‌بندی به صورت دستی و توسط یک پزشک متخصص صورت می‌گیرد که به عنوان مثال برای قطعه‌بندی تصویر سی‌تی اسکن سر و گردن یک بیمار و مشخص کردن ساختارهای در ریسک، به صورت تقریبی برای یک متخصص آموزش دیده، 3 ساعت زمان خواهد برد \مرجع{qazi2011auto} که این عملیات در روند درمان، به علت تغییر شکل تومور باید چندین مرتبه تکرار گردد \مرجع{stapleford2010evaluation}. مشکلات دیگری علاوه بر زمان‌بر بودن در قطعه‌بندی به صورت دستی وجود دارد، از جمله: امکان بروز خطای انسانی به علت مشغله‌ی کاری یا خستگی، خسته‌کننده بودن روند قطعه‌بندی و سلیقه‌ی شخص متخصص در طیقه‌بندی بافت‌ها در بین بیماران مختلف، می‌توان اشاره کرد. بنابراین ارائه‌ی یک الگوریتم خودکار\LTRfootnote{َAutomated} یا نیمه-خودکار\LTRfootnote{َSemi-Automated} برای تسریع روند قطعه‌بندی و نیز حل مشکلات ذکر شده ضروری است. 


\قسمت{ادبیات موضوع}
هدف از قطعه‌بندی در تصاویر، طبقه بندی پیکسل‌ها\LTRfootnote{Pixels}  در تصاویر دوبعدی\LTRfootnote{2D images}  یا واکسل‌ها\LTRfootnote{Voxels}  در تصاویر سه‌بعدی\LTRfootnote{3D images}  به یک طبقه‌ی خاص است که هر طبقه یک ساختار و یا ارگان خاصی را تعریف می‌کند. باوجود پیشرفت‌های چشم‌گیر در حوزه‌های پردازش تصویر\LTRfootnote{Image Processing}   و بینایی کامپیوتر\LTRfootnote{Computer Vision}، قطعه‌بندی خودکار و نیمه-خودکار، هنوز یک عملیات پرچالش در زمینه‌های تحلیل تصاویر پزشکی\LTRfootnote{Medical Images Analysis}  ، پردازش تصاویر طبیعی \LTRfootnote{Natural Images Processing} و موتورهای جستجوگر\LTRfootnote{Search Engines} تصویر است \مرجع{lateef2019survey, kayalibay2017cnn, arabi2017comparison}. 

برای قطعه‌بندی تصاویر پزشکی الگوریتم‌های خودکار و نیمه-خودکار فراوانی در دهه‌های گذشته برای تسریع در روند تشخیص و درمان ارائه گردید. از جمله‌ی این روش‌ها می‌توان به موارد زیر اشاره کرد:
 \شروع{فقرات}
 \فقره \مهم{الگوریتم‌های مبتنی بر شدت پیکسل‌ها (واکسل‌ها):} در این نوع الگوریتم‌ها معمولا چندین رویکرد در نظر گرفته می‌شود که شامل آستانه‌گذاری‌های سراسری و یا محلی\LTRfootnote{Global and/or Local Theresholding}  بر روی شدت پیکسل‌ها و الگوریتم مبتنی بر رشد ناحیه‌ای\LTRfootnote{Region Growing}  در یک منطقه از تصویر با استفاده از یک بذر\LTRfootnote{Seed}  اولیه که کاربر می‌دهد.
 
  \فقره \مهم{الگوریتم‌های خوشه‌بندی:}\LTRfootnote{Clustering} این نوع الگوریتم‌ها بر پایه‌ی روش‌های غیرنظارتی\LTRfootnote{Unsupervised} هستند و سعی در خوشه‌بندی ساختارهای موجود در تصویر با توجه به شدت‌ پیکسل‌ها دارند. از جمله‌ی این روش‌ها می‌توان به روش‌های  K-means و  \lr{Fuzzy Clustering C-means} اشاره کرد.
  
  \فقره \مهم{الگوریتم‌های مبتنی بر مدل‌های شکل‌پذیر:}\LTRfootnote{Active Contours} در این روش قطعه‌بندی، یک کانتور\LTRfootnote{Contour} اولیه توسط کاربر تعیین می‌شود و با کمینه یک تابع‌ هزینه، کانتور حرکت می‌کند و بر روی مرز ساختار مورد نظر قرار می‌گیرد. از جمله‌ی این روش‌ها می‌توان به الگوریتم Snake و Level-Set اشاره کرد.
 \پایان{فقرات}
 علاوه بر روش‌های ذکر شده در بالا، روش‌های دیگری مانند روش‌های قطعه‌بندی مبتنی بر اطلس\LTRfootnote{Atlas based Segmentation}، روش‌های مبتنی بر طبقه‌بندی\LTRfootnote{Classification} و ... نیز وجود دارد که در فصل‌های بعد به تفصیل مورد مطالعه و مرور قرار می‌گرند. 
 
 با ظهور و گسترش روش‌های یادگیری عمیق در بحث طبقه‌بندی تصاویر طبیعی و بدست آمدن نتایج بسیار خیره کننده‌ی این حوزه با استفاده از شبکه‌های کانوولوشنی عمیق\LTRfootnote{Deep Convolutional Neural Networks} بسیاری از توجهات برای استفاده از این ابزار در قطعه‌بندی تصاویر با استفاده از  روش‌های یادگیری عمیق، جلب شد. با به‌کار گرفتن ابزارهای یادگیری ماشین و یادگیری عمیق در بحث قطعه‌بندی نتایج بسیار بهتری نسبت به سایر الگوریتم‌های موجود بدست آمده است \مرجع{ren2018interleaved}. 

\قسمت{اهداف تحقیق}

در این پژوهش قصد داریم با استفاده از ابزارهای یادگیری ماشین و یادگیری عمیق روش‌هایی ارایه دهیم که علاوه بر بالا بردن دقت قطعه‌بندی تصاویر پزشکی، امکان قطعه‌بندی چند ساختار به صورت همزمان  (ساختارهای در ریسک) وجود داشته باشد. 

همانطور که پیش‌تر اشاره گردید، برای رادیوتراپی لازم است قطعه‌بندی تومور و ساختارهای در ریسک موجود باشد. عملیات قطعه‌بندی تومور به علت تغییر اندازه، موقعیت و شکل آن یک عملیات بسیار چالشی برای الگوریتم‌های تمام خودکار است. در این مطالعه، ما روشی ارایه داده‌ایم تا قطعه‌بندی تومور با دقت بالاتری با استفاده از روش‌های یادگیری عمیق صورت پذیرد و در برابر چالش‌های تغییر اندازه، موقعیت و شکل تومور قرار گیریم.
\قسمت{ساختار پایان‌نامه}

این پایان‌نامه شامل پنج فصل است. 
فصل دوم دربرگیرنده‌ی تعاریف اولیه‌ی مرتبط با پایان‌نامه است. 
در فصل سوم مسئله‌ی قطعه‌بندی تصاویر پزشکی و کارهای مرتبطی که در این زمینه انجام شده به تفصیل بیان می‌گردد. 
در فصل چهارم نتایج جدیدی که در این پایان‌نامه به دست آمده ارائه می‌گردد. در این فصل، روش‌های ارایه شده برای قطعه‌بندی ساختارهای در ریسک و تومور توضیح داده خواهد شد.
فصل پنجم به نتیجه‌گیری و پیشنهادهایی برای کارهای آتی خواهد پرداخت.
