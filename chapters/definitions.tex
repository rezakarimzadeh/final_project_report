
\فصل{مفاهیم اولیه}

در این فصل به معرفی حوزه‌ها و مفاهیم اولیه‌ی مرتبط با این پژوهش پرداخته می‌شود. در ابتدا، انواع تصویربرداری پزشکی معرفی و بررسی خواهد شد و در ادامه به تعریف حوزه‌های مرتبط با پردازش تصاویر پزشکی و بینایی کامپیوتر پرداخته می‌شود.

%----------------------------- مقدمه ----------------------------------


\قسمت{تصویربرداری پزشکی}
زمینه‌ی تصویربرداری پزشکی و سیستم‌های تصویرگر، یک زمینه‌ی بسیار گسترده و به نوبه‌ی خود پیچیده است که با استفاده از انواع پرتوهای ایکس، فراصوت، گاما، امواج الکترومغناطیس و ... با تاباندن به بدن بیمار و بازسازی تصویر از روی پرتوهای دریافتی، صورت می‌‌گیرد\زیرنویس{البته در تصویربرداری هسته‌ای، برعکس سایر روش‌های تصویربرداری، منبع تابش پرتو درون بدن بیمار قرار می‌گیرد و با آشکارسازی پرتوهای دریافتی تصویر نهایی تشکیل می‌گردد.}. تصویربرداری پزشکی کمک بسیار زیادی به متخصصین حوزه‌ی سلامت در روند تشخیص و درمان صورت داده است، به صورت کلی می‌توان روش‌های تصویر برداری پزشکی را به صورت زیر خلاصه کرد\مرجع{elangovan2016medical}:
\شروع{فقرات}
\فقره رادیوگرافی\LTRfootnote{Radiography}: که شامل تصویربردای‌های بر مبنای پرتوی ایکس است که از مهمترین آن‌ها می‌توان به فلوروسکوپی\LTRfootnote{Fluoroscopy} و رادیوگرافی‌های پروجکشنال\LTRfootnote{Projectional radiographs} اشاره کرد.

\فقره تصویربرداری هسته‌ای\LTRfootnote{Nuclear Imaging}: که با تزریق ایزوتوپ‌های خاص و ذره‌های پرانرژی گسیل شده از مواد رادیواکتیو به بدن بیمار و آشکارسازی پرتوهای دریافتی، تصویر نهایی ساخته می‌شود. از جمله روش‌های تصویربرداری در این حوزه میتوان به PET\LTRfootnote{Positron emission tomography }  و SPECT\LTRfootnote{single-photon emission computerized tomography}  اشاره کرد.

\فقره تصویربرداری فراصوت\LTRfootnote{Ultrasound}: در این نوع از تصویربرداری از کریستال‌های پیزوالکتریک\LTRfootnote{Piezoelectric} برای تولید صوت با فرکانس بالا استفاده می‌شود این صوت به بافت بدن تابانده می‌شود و بازسازی تصویر از روی صوت بازگشتی صورت می‌گیرد. 

\فقره تصویربرداری توموگرافی\LTRfootnote{Tomography}: در تصویربرداری توموگرافی هدف ساخت یک تصویر سه‌بعدی است برای این منظور از لایه‌های مختلف یک شئ بدون آن که بریده شود تصویربرداری صورت می‌گیرد و این لایه‌ها برروی یکدیگر انباشت می‌شوند و در نهایت تصویر سه‌بعدی نهایی ساخته می‌شود. از جمله روش‌های موجود در این حوزه می‌توان به تصویربرداری سی‌تی اسکن و تصویربرداری بر مبنای تشدید مغناطیسی (ام‌آرآی)\LTRfootnote{Magnetic Resonance Imaging (MRI)} اشاره نمود.

\فقره انواع دیگر تصویربرداری مانند: تصویربرداری فوتوآکوستیک\LTRfootnote{Photoacoustic imaging}، تصویربرداری حرارتی\LTRfootnote{Thermography} و ... نیز وجود دارد که به دلیل کاربردهای کیلینیکی کمتر از شرح آن‌ها خودداری می‌شود.
\پایان{فقرات}

در عملیات قطعه‌بندی تصاویر پزشکی به دلیل آن که قطعه‌بندی به طور معمول از روی تصاویر توموگرافی سه‌بعدی صورت می‌گیرد، در ادامه به شرح و بسط بیشتر روش‌های تصویربرداری سی‌تی اسکن و ام‌آرآی پرداخته می‌شود. 
 
\زیرقسمت{تصویربرداری سی‌تی اسکن}

تصویربرداری سی‌تی اسکن یک تکنیک تصویربرداری پزشکی است که در رادیولوژی برای استخراج اطلاعات از بدن به صورت غیرتهاجمی استفاده می‌شود و روند تشخیص را سرعت می‌بخشد. برخلاف دستگاه‌های معمول تصویربرداری پرتوی ایکس، که از یک منبع ثابت پرتوی ایکس استفاده می‌کنند، در سی‌تی اسکن از یک منبع متحرک مجهز به موتور استفاده می‌شود که حول گانتری\زیرنویس{Gantry به محفظه‌ی سیلندری شکل دستگاه سی‌تی اسکن گفته می‌شود که تیوب پرتوی ایکس درون آن قرار می‌گیرد.} دستگاه قابلیت چرخش دارد. در طی تصویربرداری، بیمار بر روی یک تخت قرار می‌گیرد و به آهستگی به داخل گانتری وارد می‌شود؛ در همین حین منبع پرتوی ایکس درون گانتری دور بدن بیمار می‌چرخد و باریکه‌ی پرتوهای اشعه‌ی ایکس از بدن بیمار عبور می‌کند. در سی‌تی اسکن  از آشکارسازهای دیجیتال پرتوی ایکس استفاده می‌شود که دقیقا در مقابل منبع پرتوی‌ ایکس قرار گرفته‌اند و با آشکار سازی اشعه‌ی عبوری از بدن بیمار، یک سیگنال به کامپیوتر ارسال می‌شود. شکل ~\رجوع{شکل:سی‌تی} شمای کلی یک دستگاه سی‌تی اسکن را نشان می‌دهد.

\شروع{شکل}[hb]
\centerimg{02ctscan.jpg}{7cm}
\شرح{دستگاه سی‌تی اسکن \مرجع{ct}}
\برچسب{شکل:سی‌تی}
\پایان{شکل}

هر مرتبه که منبع پرتوی ایکس یک چرخش کامل را انجام می‌دهد دستگاه سی‌تی اسکن از تکنیک‌های پیچیده‌ی ریاضیاتی برای ساخت تصویر دو بعدی برای هر لایه از بدن بیمار از روی سیگنال‌های دریافتی انجام می‌دهد. ضخامت این لایه‌ها بستگی به نوع دستگاه سی‌تی اسکن دارد اما به طور معمول بین یک تا ده میلی‌متر از بافت بدن برای هر لایه است. وقتی بازسازی یک لایه به اتمام رسید برروی لایه‌های قبلی انباشته می‌شود و در نهایت یک تصویر سه‌بعدی ساخته می‌شود.

هر لایه از تصویر ساخته‌شده امکان نمایش به صورت مجزا و یا به صورت انباشه شده و سه‌بعدی را دارد که قابلیت نمایش، اسکلت، ساختارها و بافت‌های بدن و همچنین ناهنجاری‌های ایجاد شده در بدن را دارا می‌باشد و امکان تشخیص را برای پزشک مهیا می‌سازد. استفاده از تصاویر سی‌تی در روند تشخیص و درمان فواید زیادی دارد از جمله، توانایی چرخش و جابجایی بین لایه‌های مختلف تصویر که امکان مکان‌یابی موقعیت دقیق ناهنجاری را فراهم می‌سازد. شکل~\رجوع{شکل:نمونه سی‌تی} یک نمونه تصویر از ناحیه‌ی شکم را نشان می‌دهد که در آن ساختارها و بافت‌های مختلف به راحتی قابل تفکیک است.
\شروع{شکل}[hb]
\centerimg{02ctexp.png}{7cm}
\شرح{نمونه‌ی تصویر سی‌تی اسکن از ناحیه‌ی شکم در سه نمای مختلف \مرجع{ctsamp}}
\برچسب{شکل:نمونه سی‌تی}
\پایان{شکل}

از سی‌تی اسکن می‌توان در تشخیص بیماری و آسیب نواحی مختلف بدن استفاده نمود به عنوان مثال از سی‌تی در تشخیص تومور و غدد سرطانی و سایر ناهنجاری‌ها در نواحی شکم، سر و گردن و قفسه‌ی سینه بسیار استفاده می‌شود \مرجع{ctexplain}.  

\زیرقسمت{تصویربرداری ام‌آزآی}


