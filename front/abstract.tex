
% -------------------------------------------------------
%  Abstract
% -------------------------------------------------------


\pagestyle{empty}

\شروع{وسط‌چین}
\مهم{چکیده}
\پایان{وسط‌چین}
\بدون‌تورفتگی

برای رادیوتراپی و از بین بردن بافت‌های سرطانی لازم است پیش از درمان و تاباندن پرتوی پرانرژی، مکان تومور و ساختارهای آسیپ پذیر اطراف تومور با دقت بالایی مشخص شوند. برای این کار لازم است تصاویر دریافت شده از بیمار قطعه‌بندی شوند. این کار معمولا به صورت دستی انجام می‌شود که علاوه بر زمان‌بر بودن هزینه‌ی بسیار زیادی نیز دارد. 

روش‌های متنوعی برای قطعه‌بندی این تصاویر به صورت خودکار و نیمه-خودکار ارایه شده‌است که در میان آن‌ها روش‌های مبتنی بر یادگیری ماشین و یادگیری عمیق دقت بسیار بالاتری نسبت به سایر روش‌ها نشان داده‌اند. این روش‌ها علی‌رغم این برتری مشکلاتی از قبیل هزینه‌ها محاسباتی بالا، عدم توانایی در یادگیری شکل و ساختار در قطعه‌بندی بافت مورد نظر و عدم توانایی در قطعه‌بندی تومور به علت متغییر بودن اندازه، موقعیت و شکل آن را دارا می‌باشند. 

در این مطالعه، برای حل مشکلات اشاره‌شده روش‌هایی ارائه گردیده است که با انتقال دانش از یک مدل پیچیده به یک مدل ساده‌تر و نیز ارائه‌ی یک چارچوب بر اساس باخورد خطا، بدون افزایش پیچیدگی و هزینه‌های محاسباتی در یک مدل ساده، دقت آن افزایش یابد. همچنین برای حل مشکل یادگیری شکل و ساختار برای قطعه‌بندی ساختارهای در ریسک، یک تابع هزینه بر اساس شکل پیشنهاد گردیده است که بر اساس فضای شکل تعریف شده، معتبر یا نامعتبر بودن شکل پیش‌بینی شده برای یک ساختار سه‌بعدی را ارزیابی می‌کند. در نهایت برای قطعه‌بندی تومور یک چارچوب بر اساس سازوکار توجه پیشنهاد شده است که از دو ماژول توجه و قطعه‌بند تشکیل گردیده است. ماژول توجه وظیفه‌ی یافتن موقعیت تومور و ماژول قطعه‌بند با توجه به اطلاعات ماژول توجه، قطعه‌بندی نهایی را انجام می‌دهد. ارزیابی‌های نهایی بر روی مجموعه دادگان مختلف نشان داد، روش‌های پیشنهادی دقت قطعه‌بندی را افزایش خواهند داد.


\پرش‌بلند
\بدون‌تورفتگی \مهم{کلیدواژه‌ها}: 
قطعه‌بندی ساختار‌های در ریسک، قطعه‌بندی تومور، یادگیری عمیق، شبکه‌های عمیق کانوولوشنی
\صفحه‌جدید
