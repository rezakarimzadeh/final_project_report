
% -------------------------------------------------------
%  English Abstract
% -------------------------------------------------------


\pagestyle{empty}

\begin{latin}

\begin{center}
\textbf{Abstract}
\end{center}
\baselineskip=.8\baselineskip

For radiotherapy and removal of cancerous tissues, it is necessary to determine the location of the tumor and the vulnerable structures around the tumor before treating and irradiating the high-energy beam. To do this, the images received from the patient need to be segmented. This is usually done manually, which is not only time consuming but also very expensive.

Various methods for segmenting these images are presented automatically and semi-automatically, among which methods based on machine learning and deep learning have shown much higher accuracy than other methods. Despite this superiority, these methods have problems such as high computational costs, inability to learn the shape and structure of the tissue and the inability to segment the tumor due to the variability of its size, position and shape.

In this study, to solve the mentioned problems, methods have been proposed that increase its accuracy by transferring knowledge from a complex model to a simpler model and also providing a framework based on error feedback, without increasing complexity and computational costs in a simple model. Also, to solve the shape and structure learning problem for segmenting organs at risk, a shape-based cost function is proposed, which evaluates the validity or invalidity of the predicted shape for a three-dimensional structure based on the defined shape space. Finally, a framework based on the attention mechanism has been proposed for tumor segmentation, which consists of two modules of attention and segmentation. Attention module The task of finding the position of the tumor and the segmentation module performs the final segmentation according to the information of the module of attention. Final evaluations of the various datasets showed that the proposed methods would increase the accuracy of segmentation. For example, in using the shape-based cost function for organ segmentation, the Dice index for test data was increased from $0.74$ to $0.81$, and from $0.68$ to $0.79$, using the attention module in tumor segmentation.

\bigskip\noindent\textbf{Keywords}:
Organs at Risk Segmentation, Tumor Segmentation, Deep Learning, Deep Convolutional Neural Networks

\end{latin}

\newpage
